%translator Russkov, date 27.04.13

\setcounter{Equation}{0} \setcounter{Figure}{0}
\Work
{Interaction of {$\boldsymbol\gamma$-rays}\\ with scintillation detectors}
{Interaction of {$\boldsymbol\gamma$-rays}\\ with scintillation detectors}
{Response of organic (plastic) and inorganic ($\mathrm{NaI}(\mathrm{Tl})$) scintillators to monochromatic $\gamma$-radiation and the contribution of photoelectric absorption and the Compton scattering to the distribution of scintillation amplitudes are studied.}

The most effective way of measuring the energy of $\gamma$-quanta is to determine the energy of electrons produced by interaction of the radiation with atoms. The electrons undergo a large number of inelastic collisions with atoms of the medium, these processes are accompanied by ionization and excitation of the atoms. As a result the kinetic energy of an electron transforms into the energy of thermal motion of matter. At intermediate stages, e.g. transition of excited molecule or atom into ground state and recombination of electric charges in a medium, the optical quanta of various wavelengths inherent to a given material are emitted.

The ability of some materials to emit light in response to ionizing radiation was discovered at the beginning of the XX-th century, the corresponding flashes were called \textit{scintillations}. \textit{Scintillator} is a material which being subjected to charged particles or a long-wave electromagnetic radiation emits photons of visible or ultraviolet range. A material which can serve as scintillator has the following properties: firstly, the probability of photon emission by atoms and molecules from an excited state is high, and, secondly, the probability of absorption of the emitted photons is low. In other words, the scintillator emission spectrum must be displaced relative to the absorption spectrum; otherwise a photon emitted by an atom or a molecule will be immediately absorbed by another atom or molecule (the resonance absorption of light occurs).

The energy of scintillator light flashes is transformed into electrical pulses by a photoelectric multiplier tube (PMT). A photon striking the PMT photocathode knocks out electrons due to the photoelectric effect. The electrons are then accelerated by the external field and strike PMT dynodes where the secondary electron emission takes place; eventually the number of electrons reaching the PMT anode increases by tens and hundreds thousand times. The generated pulse of electric current is then registered by an electronic circuit. Usually the pulse amplitude is directly proportional to the absorbed energy which allows one to determine the energy of $\gamma$-quantum. Processes of light flashes formation in different scintillators and the principle of PMT operation are discussed in detail in Appendices~II and~III.

Three characteristics of a scintillation detector are of special importance for nuclear radiation spectrometry:

1)\;the conversion effectiveness, i.e. the ratio of the energy of light flash to the energy absorbed by the scintillator;

2)\; the time of luminescence which characterizes the flash duration and therefore restricts the temporal resolution of spectrometer;

3)\; the energy resolution.

The energy resolution is, of course, the main characteristics of a scintillator. In principle a scintillation counter must possess a high energy resolution. Indeed, in most scintillators the light flash intensity, at least in a specified range, is proportional to the energy loss; the effectiveness of light condensing on the PMT photocathode is determined by its geometry which is independent of scintillation intensity; the photoelectron yield is determined solely by the photocathode properties and, finally, the PMT gain in the normal operation mode is constant.

However in a real scintillation counter there is a host of phenomena leading to degradation of its energy resolution. The current pulse on the PMT anode fluctuates due to fluctuation of the number of photons emitted by scintillator, the probabilistic nature of the photoelectric effect at the cathode, and fluctuation of the PMT gain. A radiation leakage from scintillator crystal can be significant as well. It is due to the fact that electrons can be produced in the vicinity of the crystal surface and escape losing part of their energy outside the scintillator. The greater the crystal conversion effectiveness, the better is its energy resolution.

As it has already been mentioned scintillation flashes are due to interaction of charged particles with a medium. In the case of electromagnetic radiation these particles are electrons produced by three different processes: the photoelectric effect, the Compton scattering, and the pair production. All three ways of interaction of $\gamma$-quanta with a material generate fast electrons able to ionize the material and to create light flashes (scintillations) which number is somehow related to the energy of $\gamma$-quanta.

All three processes are discussed in detail in the introduction to this section. Therefore below we mention only the characteristic dependence of a process on the energy of $\gamma$-quantum and on the atomic number of the (scintillator) medium through which the photons pass.
\vspace{1ex}

\textit{The photoelectric effect.}
Photoelectric absorption can take place, in principle, in any atomic shell; but if the energy of \mbox{$\gamma$-quantum} exceeds the binding energy of electron on $K$-shell, the photoelectric effect  mainly consists in knocking out an $K$-electron. When the energy of $\gamma$-quantum increases (far from the $K$-absorption band edge), the effective cross-section of the photoelectric effect on the $K$-shell decreases as $\hbar\omega^{-3{.}5}$ and at high energy ($\hbar\omega\gg mc^2$) it is proportional to $1/(\hbar \omega )$. The probability of photoelectric absorption strongly depends on the atomic number of the material, namely, as $Z^5$.
\vspace{1ex}

\textit{The Compton scattering.}
The Compton scattering cross-section at low energy of incident $\gamma$-quantum ($\hbar\omega\ll mc^2$) remains almost constant, then at $100\;\keV$ and above it decreases as $1/(\hbar\omega)$, and at the energy about $10\;\MeV$ it attains $10\%$ of its maximal value. The effective Compton scattering cross-section is proportional to the atomic number $Z$.
\vspace{1ex}

\textit{The electron-positron pair production.}
It is a threshold process which begins only at the energy $E_{\gamma}>2mc^{2}=1{.}02\;\MeV$. The effective pair production cross-section is proportional to $Z^2$; when the energy of \mbox{$\gamma$-quanta} increases the cross-section grows very fast at first and then slows down approaching a constant at high energy.

Now consider the physical processes responsible for the most characteristic areas of the instrumental line of a scintillation detector spectrometer.

\vspace{4pt}
1). \textit{The complete absorption of the energy of primary \mbox{$\gamma$-quantum} in a sensitive detector volume.} Such absorption occurs due to the photoelectric effect, multiple Compton scattering, and a pair production in the detector depth in which the energy of photo- and Compton electrons, electron-positron pairs, annihilation quanta, and a characteristic radiation of the scintillator material is completely absorbed by the crystal. This process corresponds to the so-called \textit{complete absorption peak}.

\vspace{4pt}
2). \textit{An incomplete absorption in which the energy of \mbox{$\gamma$-quantum} is partially carried away from the scintillator.} First of all this happens due to the Compton scattering in which the energy of scattered $\gamma$-quantum is
$$
  E_{\gamma}'=\frac{mc^2}{\frac{mc^2}{E_{\gamma}}+1-\cos\theta}.
  \eqMark{5_4_1}
$$

The minimal energy of the scattered quantum (corresponding to $\theta = \pi$) equals
$$
  E_{\gamma ,\mathrm{\min}}'=\frac{mc^2}{\frac{mc^2}{E_{\gamma}}+2}.
  \eqMark{5_4_2}
$$

The distribution of pulse amplitudes corresponding to absorption of the recoil energy of scattered electron turns out to be continuous (the Compton continuum) to the Compton edge $E_{\gamma}-E_{\gamma ,\mathrm{\min}}'$.

The energy distribution of the recoil electrons generated by the primary Compton scattering is shown in \refFigure{5_4_1} to explain irregularities in the Compton continuum. The distribution is continuous up to the maximum located at the energy
$$
  E_{e}=\frac{E_{\gamma}}{1+\frac{mc^2}{2E_{\gamma}}}.
  \eqMark{5_4_3}
$$

One can see that the edge of Compton scattering is displaced from $E_{\gamma}$ approximately by $mc^2/2=0{.}25\;\MeV$ (for $E_{\gamma}\gg mc^2$) and is specified by a sharp maximum at $E_e$.

The $\gamma$-quanta scattered on the PMT window, on the walls of scintillation crystal, etc. strike the scintillator in the range of angles $90\degree\div 180\degree$, forming the so-called backscatter peak. Its
%
\fFigure{Energy distribution of the Compton scattering electron for the $\gamma$-quantum energy of $5,10$ and $15\;\meV$}5_4_1 {4.9cm}{2.7cm}{pic/L05_4_01.eps}
%
energy is about $mc^2/2$, and it is a supplemental peak to the Compton maximum, i.e.
$$
  E_e+E\sub{back}=E_{\gamma}.
  \eqMark{5_4_4}
$$

The process of pair production in which an annihilation quantum escapes from a scintillator also results in a partial absorption of the energy of primary \mbox{$\gamma$-quantum}. If both annihilating quanta escape from the scintillator the absorbed energy is $E_{\gamma}-2mc^2$ (the so-called double-escape peak). If one annihilation quantum loses its energy in the crystal, this process contributes to the single-escape peak at $E_{\gamma}-mc^2$. The pair production can occur in the scintillator surroundings and the annihilation quanta with energy $0{.}511\;\MeV$ can be absorbed by the scintillator.

%
\hFigure{Instrumental line of $\gamma$-radiation scintillation spectrometer: $\textit{1}$~is the complete absorption peak (the photoelectric effect), $\textit{2}$~is the edge of the Compton  continuum, \textit{3} is the single-escape peak, \textit{4}~is the double-escape peak, $\textit{5}$~is the annihilation peak, $\textit{6}$~is the backscatter peak, $\textit{7}$~is the Compton continuum, and $\textit{8}$~is the peak of characteristic radiation of shield material}5_4_2
{8.1cm}{4.2cm}{pic/L05_4_02.eps}
%

A hypothetical instrumental spectrum of \mbox{$\gamma$-radiation} scintillation spectrometer is shown in \refFigure{5_4_2} for illustration purposes.

Relative intensity of different spectrum areas  significantly depends on the scintillator material and on its dimensions. As the crystal size and, hence, a path of \mbox{$\gamma$-quantum} in crystal increase, the probability of multiple Compton scattering increases significantly. Besides, since the resolution time of crystal is greater than the time of collisions, the processes sum up. Therefore in  relatively large occurrences the total energy of \mbox{$\gamma$-quantum} remains in the crystal, and the corresponding pulse contributes to the complete absorption peak. Such pulse <<pumping>> from the continuous 
%
\hFigure{Gamma spectra of isotope $^137\mathrm{Cs}$, obtained with crystal $\mathrm{NaI}(\mathrm{Tl})$ of following dimensions: $\textit{1}$~---
$\varnothing30 \times10$ mm; \textit{2}~--- $\varnothing30\times20$ mm; \textit{3}~--- $\varnothing40\times40$ mm; \textit{4}~--- $\varnothing120\times100$ mm}5_4_3 {9cm}{9.1cm}{pic/L05_4_03.eps}
%
distribution to the complete absorption peak proceeds more effectively for a larger crystal. The absence of the continuous Compton distribution would be ideal in analyzing a complicated spectrum. However  the complete suppression of the distribution turns out to be impossible even for a very large crystal. As the crystal dimensions increase, the backscatter and edge effects though reduced still persist. 

In~\refFigure{5_4_3} one can see the amplitude distributions of monochromatic \mbox{$\gamma$-radiation} of energy $662\;\keV$ emitted by nuclei of $^{137}\mathrm{Cs}$. The distributions are measured by means of crystals of $\mathrm{NaI}(\mathrm{Tl})$ of different size. The distributions are normalized by the photopeak to facilitate comparison.

The photoelectric absorption and the pair production contribute to the maxima of amplitude distribution, whereas the Compton scattering forms the continuous background. The photoelectric absorption contribute to the complete absorption peak since the primary energy is completely transferred to electron and X-ray photon. The pair production forms three peaks in the amplitude distribution: the first one is the complete absorption peak, when both annihilation \mbox{$\gamma$-quanta} are absorbed in a scintillator crystal; the second one corresponds to an escape of a single annihilation 
%
\fFigure{Response functions of organic and inorganic crystals: \textit{a}~is crystal of $\mathrm{NaI}(\mathrm{Tl})$ and $\gamma$-quanta with energy $1;\meV$, \textit{b}~is stilbene crystal and $\gamma$-quanta with energy $4.5;\meV$}5_4_4
{5.3cm}{4.9cm}{pic/L05_4_04.eps}
%
quantum from the crystal, and the third one corresponds to an escape of both annihilation quanta. Thus the amplitude distributions corresponding to detection of \mbox{$\gamma$-quanta} by inorganic crystals turns out to be rather complicated. The diagram~in \refFigure{5_4_4} shows the response functions typical of two sorts of crystal, namely, inorganic ($\mathrm{NaI}(\mathrm{Tl})$ and organic (stilbene and polystyrene). The inorganic crystals contain elements with high atomic numbers, so in these crystals the complete absorption peak is prominent due to the photoelectric effect. The response function of organic crystals to \mbox{$\gamma$-quanta} is represented by a continuous distribution due to the Compton scattering. In these crystals the probability of photoelectric effect  turns out to be approximately $20$ times less than the probability of Compton scattering even for \mbox{$\gamma$-quanta} of $50\;\keV$. The edge of the Compton continuum corresponds to backscattered \mbox{$\gamma$-quanta}.

\vspace{8pt}
\textbf{\so{Experimental installation}}\vspace{4pt}

In this experiment the following cylindric crystals are used. Two are made of $\mathrm{NaI}(\mathrm{Tl})$: one is $25\;\mm$ in diameter and $40\;\mm$ in height and the other one is $80\;\mm$ in diameter and $80\;\mm$ in height. The third one is a polystyrene crystal $40\;\mm$ in diameter and $40\;\mm$ in height. The output window of a crystal is in optical contact with the photocathode of a PMT operating in a spectrometric mode. The signal from the PMT anode is applied to an amplitude analyzer via an amplifier. Each scintillator is assembled in a single unit with a PMT; the unit can be placed against the beam of \mbox{$\gamma$-quanta} emitted by a radioactive source enclosed in a thick lead container. The radiation comes out via a small collimator hole.

The single-channel amplitude analyzer used in the installation is designed to analyze pulses in the range from $0$ to $10\;\V$. However if the crystal $\mathrm{NaI}(\mathrm{Tl})$ is irradiated by \mbox{$\gamma$-quanta} with an energy of $\sim 1\;\MeV$ the maximal amplitude at the PMT output does not exceed tenths of volt. Therefore the PMT output signal must be preliminarily amplified by a linear amplifier, although the maximal amplitude must be less than $10\;\V$.
\vspace{1ex}

\textbf{\so{Directions}}\vspace{5pt}

\begin{Enumerate}{tab}
\Item. Turn on the installation and let it warm up for $5\div10$ minutes.

\Item. Set the operation mode indicated on the installation.

\Item. Make sure that the installation <<feels>> $\gamma$-rays. To this end bring the detector unit with a crystal of $\mathrm{NaI}(\mathrm{Tl})$ to the collimator hole of the $\gamma$-source $^{137}\mathrm{Cs}$ emitting quanta with energy $E_{\gamma}=662\;\keV$. Set an arbitrary threshold on the amplitude analyzer and turn on the scaler. The counter must start counting the pulses. Remove the detector from the source and verify that the count rate abruptly decreases. Then proceed to the main measurement.

\Item. Measure the count rate versus an analyzer threshold by using the $^{137}\mathrm{Cs}$ source of \mbox{$\gamma$-quanta}. Since the absorption photopeak corresponds to a <<high>> threshold, the measurement should begin from the highest threshold. Write down the obtained values, and then plot them on a sheet of graph paper versus a threshold.

\Item. Determine the photopeak location (complete absorption peak) and graduate the abscissa both in megaelectronvolts and in volts (a threshold scale) assuming the direct proportionality . 

\Item. To achieve a higher accuracy repeat the measurement in the vicinity of the complete absorption peak by increasing the time of measurement. 

\Item. Carry out a similar measurement using the other crystal of $\mathrm{NaI}(\mathrm{Tl})$ and the plastic scintillator.

\Item. Using the spectra obtained calculate the number of events (in percent) resulted in the complete energy loss for the crystals of different size. Estimate the relative probability of photoelectric absorption and the Compton scattering in various crystals. Using the energy of incident \mbox{$\gamma$-quanta} calculate the position of the edge of Compton distribution and compare it with the experimental value for all crystals. 

\Item. Estimate the spectrometer energy resolution.

\Item. When the experiment is over close the collimators of $\gamma$-sources by lead caps and turn off the installation.
\end{Enumerate}

\begin{center}\so{\textsf{\small LITERATURE}}\end{center}
{\small

1. \textit{Shirokov\;Yu.\;M., Yudin\;N.\;P.} Nuclear physics.\,---\,M.: Science, $1980$. Ch.\;VI, \textsection\;$6$; ch.\;IX,\textsection\;$4$.

2. \textit{Muhin\;K.\;N.} Introduction to nuclear physics.\,---\,�.: Atompress, $1965$. ��.\;II, \textsection\;$11$.

3. \textit{Tsipenyuk\;Yu.\;M.} Principals and methods of nuclear physics.\,---\,�.: Energyatompress, $1993$. \textsection\;$5.6$.
} 



