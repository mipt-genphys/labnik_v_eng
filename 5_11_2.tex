%translator Svintsov, date 17.02.13

\setcounter{Equation}{0} \setcounter{Figure}{0}
\Work
{Photoconductivity of semiconductors }
{Photoconductivity of semiconductors}
{Intrinsic and extrinsic photoconductivity are studied. The obtained spectral dependence of photocurrent makes it possible to determine the semiconductor band gap and the ionization energy of impurities.}

Semiconductors feature the ability of increasing their conductivity when exposed to light. This phenomenon is called \textit{inner photoelectric effect} or \textit{photoconductivity}.

Light absorption increases the number of charge carriers in a semiconductor. Photoconductivity is due to three types of electron transitions induced by photon absorption (see~\refFigure{11_2_1}). The first type is electron transition from the occupied (valence) band to the conduction band.
%
\fFigure{Types of transitions giving rise to semiconductor photoconductivity}11_2_1 {6.11cm}{2.44cm}{pic/L11_2_01.eps}
%
These transitions result in generation of free electrons and holes. The photoconductivity emerging due to these transitions is called \textit{intrinsic photoconductivity}. Transition of the second type occurs when photon is absorbed by atom of donor dopant, this brings electron in the conduction band and leaves a vacancy in the donor atom. 

Transition of the third type occurs when electron is transferred from the valence band to a free acceptor level. This process results in generation of free holes and electrons bound by acceptor atoms. Photoconductivity caused by two last processes is called \textit{extrinsic photoconductivity}.

\vspace{-0.5pt}
A small number of charge carriers is present in semiconductor even in the absence of light. Electrons are transferred from the valence band (and from donor levels) to the conduction band (and to acceptor levels) by thermal motion. 

Concentration of such carriers, and, therefore electrical conductivity, rapidly increases with temperature. In this case one can talk about \textit{equilibrium} charge carriers and \textit{dark} conductivity. The concentration of charge carriers is equal to the equilibrium concentration not only in complete darkness, but also when the photon energy is not high enough to induce electron transitions in the crystal. Photoconductivity emerges only when the frequency of light exceeds a threshold. The corresponding frequency is called \textit{photoemission threshold}.

\vspace{-0.5pt}
Unlike thermal energy, the energy of light is stored by semiconductor electrons and does not change the temperature of crystal lattice. Therefore, thermal equilibrium between electrons and the lattice is disturbed in the presence of light. Charge carriers due to optical ionization are \textit{non-equilibrium}.

When illumination of semiconductor crystal stops thermal equilibrium between electrons and the lattice restores. Under normal conditions the energy stored by the non-equilibrium carriers is negligible compared to the thermal energy of the lattice. The process of restoring thermal equilibrium between the lattice and electrons is recombination of non-equilibrium electrons and holes that does not significantly change the crystal temperature. Therefore, the concentration of equilibrium carriers is not changed either. Thus, switching the light on and off changes the concentration of non-equilibrium charge carriers and does not change the concentration of equilibrium ones.

The photocurrent can be measured by means of the circuit shown in~\refFigure{11_2_2}. The film specimen is connected to the circuit which includes an emf source and a resistor, the voltage across the resistor is measured by a digital voltmeter \linebreak{�$7$-$34$�}.
%
\fFigure{Circuit for measuring photocurrent}11_2_2 {6.9cm}{1.95cm}{pic/L11_2_02.eps}
%
When the specimen is illuminated the current in the circuit increases and so do the voltmeter readings. A simple circuit like this one is suitable for measurements only if the photocurrent exceeds the dark current or, at least, is of the same order. Otherwise, the experimental installation must be more sophisticated, e.g. implementing amplitude modulation of light flux. In this case it is not difficult to separate the light-associated ac component of the total current from a large (but constant!) dark current.

The dependence of photocurrent on the frequency of incident light (the spectral dependence of photocurrent) is complicated, but its characteristic feature is a cutoff at long wavelengths, i.e. an abrupt drop of the curve at low frequencies. The cutoff position shows the lowest photon energy at which charge carriers can be generated. The curve to the right of the cutoff (at higher frequencies) varies for different semiconductors. After the sharp rise the photocurrent curve can drop (a $\mathrm{CdS}$ specimen), or it can exhibit a wide plateau (like in a selenium specimen). Small additional maxima can be seen prior to the main rise corresponding to the energy at which electrons can pass from the occupied band to the conduction band. These maxima are associated with the energy levels of dopants (second- and third-type transitions in~\refFigure{11_2_1}). One can see in~\refFigure{11_2_1} that the energy of these transitions is lower than the energy of transition from the valence to the conduction band, thus their long-wavelength cutoff is to the left of the intrinsic transition cutoff.

The ratio of photocurrents caused by intrinsic and extrinsic transitions depends on dopant concentration and temperature. In a pure semiconductor the concentration of dopants is very small. Besides, photon excitation of a dopant results in generation of a single charge carrier, electron or  hole, while in intrinsic semiconductors each absorbed photon produces both electron and hole simultaneously. 

When temperature rises extrinsic photoconductivity decreases faster than intrinsic one. It could happen that even at room temperature the majority of dopant atoms are thermally ionized and the rest contribute only slightly to the photoconductivity. So, the extrinsic photoconductivity turns out to be considerably smaller than the intrinsic photoconductivity.

In this experiment the position of dopant levels and the band gap energy are determined via the characteristic rise of the photocurrent curve. 

The experiments are carried out with semiconductor films or thin plates of single crystals of $\mathrm{CdS}$ and $\mathrm{CdSe}$ (either pure or doped with copper ions). Unlike most semiconductors, these crystals have relatively wide band gap (more than $1.5\;\eV$). The acceptor levels of copper ions are far away from the occupied and the conduction bands. Under these conditions the long-wavelength cutoff of the dopant photoconductivity is in the visible light range, while for majority of semiconductors it is in the infrared range. A small dark current and a large light yield make it possible to do experiments without modulating the light flux. \vspace{1ex}

\textbf{\textso{Experimental installation}}
\vspace{2pt}

The schematic view of the experimental installation is shown in~\refFigure{11_2_3}. Light from a source~\textit{S} is focused on the entrance slit of the monochromator UM-$2$ by means of~\mbox{a lens \textit{L}} The slit is at the focus of the collimator lens~\textit{L}$_1$.

The collimated beam coming out from the lens \textit{L}$_1$ is dispersed into visible spectrum by a prism~\textit{P}. The output slit located at the focal plane of an eyepiece~\textit{2} cuts out the desired spectral band.
%
\hFigure{The installation for measuring the spectral dependence of photocurrent}11_2_3 {5.7cm}{5.6cm}{pic/L11_2_03.eps}
%
Light coming out of the output slit strikes the cell with a specimen (labeled by~\textit{S} in the figure). A regulated rectifier, which is a source of emf, is connected in series with the specimen. The voltmeter �$7$-$34$� measures the current passing through the specimen.

The spectral distribution of the photon flux at the monochromator output and the calibration curve of the monochromator are given by the diagrams on the workplace. \vspace{1ex}

\textbf{\textso{Directions}}
\vspace{2pt}

\begin{Enumerate}{tab}

\Item. At the beginning of the experiment plug the rectifier and digital voltmeter �$7$-$34$� to the mains. Let the voltmeter warm up for at least $15\;\mi$. Switch the voltmeter to the voltage measuring mode.

\Item. Check calibration of the monochromator by the yellow doublet of mercury lamp ($\lambda_{1} =5770\;\Angstrem$, $\lambda_{2} =5790\;\Angstrem$) and the yellow line of neon lamp ($\lambda_{3}=5852\;\Angstrem$). The calibration procedure is the following.

a)\;Remove the specimen unit from the output monochromator slit.

b)\;Remove the output slit and insert an eyepiece instead.

\label{focus}c)\;Switch on the mercury lamp and focus its light on the entrance slit of the monochromator using the lens~\textit{L}. The best conditions for the measurement are realized providing the monochromator optical system is completely filled with light. This is achieved if $D_{\text{L}}/b=D_{\text{L}_{1}}/F$, where $D\sub{{L}}$~is the diameter of the focusing lens, $b$~is the distance between the lens and the entrance slit, and $D_{\text{L}_{1}}$ and $F$~are the diameter and the focal length of the collimator lens. The given relationship
%
\cFigure{Choice of conditions with the best illuminance }11_2_4 {9.5cm}{2.85cm}{pic/L11_2_04.eps}
%
is illustrated by~\refFigure{11_2_4}. The relative value of the entry port of monochromator UM-$2$ is $D_{\text{L}_1}/F=1/6$.

d)\;Adjust the monochromator to the yellow doublet and compare the drum readings with the monochromator calibration curve. 

e)\;Switch off the mercury lamp; switch on the neon lamp and focus its light on the entrance slit of the monochromator.

f)\;Adjust the monochromator to the yellow line ($\lambda_{3} =5852\;\Angstrem$) and compare the drum readings with the calibration curve.
g)\;If the drum readings do not coincide with the calibration curve, shift the drum helical scale to achieve the coincidence. 
\Item. Switch on the incandescent lamp and focus the image of its filament on the entrance slit of the monochromator. Make sure that the monochromator is filled with light (see p.\,2c)).

\Item. Remove the eyepiece and insert the output slit instead.

\Item. Set the width of entrance and output slits of the monochromator to about $0.1\;\mm$. This width provides an acceptable photocurrent while maintaining the required resolution power of the spectrometer.\looseness=-1

\Item. Move the specimen unit closer to the output slit of the monochromator.

\Item. Measure the dark current of the specimen.

\Item. Measure the dependence of photocurrent on the light wavelength. The photocurrent equals the digital voltmeter readings divided by the resistance $R$ connected to the specimen circuit (the value of $R$ is indicated on the installation). Each measurement of the current should be done after a three-minute exposure. The wavelength is determined by the calibration curve. The required photocurrent is determined by subtracting the dark current from the current measured for a chosen wavelength.

\Item. Plot the spectral dependence of photocurrent for the test specimen. For this purpose it is necessary to normalize the current by the photon flux. One should use the appended chart of spectral distribution of the photon flux at the monochromator output.

\Item. Determine the ionization energy of the dopant level and the band gap of the studied semiconductor from the spectral dependence of photocurrent obtained. The type of semiconductor is indicated on the specimen unit. The position of the long wavelength cutoff approximately coincides with the middle of the corresponding rise of the photocurrent curve.

\Item. Evaluate the accuracy of the measured value of the band gap and the dopant ionization energy of semiconductor. \end{Enumerate}%

\Literat

\small 1.\:\emph{L.L.Goldin, G.N.Novikova} Introduction to Atomic Physics.\,---\,M.: Nauka, 1988. Ch.\:XIII, \S\S\:64--67.

2.\:\emph{Ch.Kittel.} Introduction to Solid-State Physics.\,---\,M.: Nauka, 1978. Ch.\:11, p.\:379--386.

3.\:\emph{R. Smith.} Semiconductors.\,---\,�.: IL, 1962. Ch.\:7, \S\S\:4,\:10; Ch.\:8, \S\:15; Ch.\:11,~\S\:7.

4.\:\emph{S.M.Ryvkin} Photoelectric Phenomena in Semiconductors.\,---\,�.: Physmatgiz, 1963. Ch.\:I, \S\S\:1--4.
\normalsize
