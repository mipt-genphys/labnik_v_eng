%translator Savrov, date 08.01.13

\let\theEquation=\oldTheEquation \let\theFigure=\oldTheFigure

\Chapter {Cosmic rays} {Cosmic rays} {Cosmic rays}

\textbf{Composition and properties of primary and secondary cosmic rays}. Cosmic rays are high energy particles, mostly, protons coming to the Earth from outer space (primary cosmic rays) and the particles generated in the Earth upper atmosphere (secondary cosmic rays) in interactions between the primary rays and atomic nuclei; the secondary cosmic rays include all known elementary particles. Cosmic rays provide a unique source of high energy particles, hence they are widely used to study the properties of matter and interactions between elementary particles. It was the study of charges and masses of the secondary cosmic rays which resulted in the discovery of positron ($1932$), muon ($1937$), $\pi$-~and $K$-mesons ($1947$), and also $\Lambda^0$-, $\Sigma^{\pm}$-hyperons. The energy of primary cosmic rays reaches $10^{20}\;\eV$, so they will long remain the unique source of particles of ultrahigh energy, since the energy of accelerated particles does not exceed $10^{12}\;\eV$. Notice, however, that modern proton colliders already reach the energy of $\Simeq 10^{17}\;\eV$ in the center-of-mass frame. 

It has long been known that the Earth is bombarded by a highly penetrating radiation. For instance, a charged electroscope gradually discharges even if it is safely shielded from the Earth natural radioactivity. Cosmic rays were discovered by V.~Hess in $1912$ who measured the ionization rate caused by the rays. The rate was increasing with altitude thereby proving that it was due to an extraterrestrial source. The composition and the properties of cosmic rays coming to the Earth from outer space were mostly studied with the aid of instruments stationed at high-altitude balloons and aerostats launched from the Earth to an altitude of up to $25\;\km$, where they sail from several hours to several days; recently it has become possible to employ artificial satellites and space stations for this purpose. 

On the one hand, the composition and the properties of cosmic rays provide us with information  about the processes occurring in stars and intergalactic space. On the other hand, this knowledge is indispensable for space exploration and for understanding the influence of cosmic rays on the processes taking place on Earth. Besides, as mentioned above, cosmic radiation is a unique natural  source of particles of high and ultrahigh energies (up to $10^{20}\;\eV$) which allows us to study the processes of particle creation and interactions.
%
\cFigure{Abundance of elements in cosmic rays (and in the Solar System, dashed curve)}7_1 {6.6cm}{8.1cm}{pic/L07_01.eps}
%
On their way to the Earth surface cosmic rays pass a thick ($\Simeq10^3\;\g/\cm^2$) layer of matter, the atmosphere, in which they undergo a complicated chain of transformations. As a result, the composition of cosmic rays reaching the surface (secondary rays) is completely different from that of the primary rays. Protons account for more than $90$\% of the net particle flux in the primary rays, helium nuclei ($\alpha$-particles) constitute about $7\%$, and about $1\%$ of the flux is due to heavier nuclei, $\mathrm{C}$, $\mathrm{N}$, and $\mathrm{O}$ (see~\refFigure{7_1}).

Abundance of elements in nature is closely related to the Universe evolution. According to the modern paradigm the age of the Universe is about $10^{10}\;$years. This is the time passed since the so-called \textit{Big Bang}, the Universe origin. At the beginning the Universe was an extremely dense and hot clump of matter which upon cooling produced elementary particles which, in turn, formed nuclei in processes of nucleosynthesis. Formation of heavy nuclei from light ones is energetically favorable only for elements lighter than the elements of the iron group which have the largest binding energy per nucleon. Therefore the elements of the iron group cannot serve as a nuclear fuel and the synthesis must stop as soon as iron has been produced. 

It is likely that the heavier elements are produced in stars in nuclear reactions of neutron capture followed by $\beta$-decay of a daughter nuclide, which increases $Z$. The heavy elements are then ejected in outer space by stellar explosion.

\vspace{-0.3pt}
Abundance of chemical elements in cosmic rays compared to the Universe average is somewhat different. On the one hand, this is due to predominant acceleration of heavy ($Z\geqslant10$) and very heavy ($Z\geqslant20$) nuclei in the sources of cosmic rays and, on the other hand, because of <<burn out>> of light nuclei ($Z=3\div5$, i.e. $\mathrm{Li}$, $\mathrm{Be}$, and $\mathrm{B}$) in the fusion reactions in stars. This <<lacuna>> in the flux of cosmic rays is partially filled due to fragmentation of heavier nuclei. There are no neutrons in the primary cosmic rays since free neutron is unstable: it decays into proton, electron, and antineutrino with the half-life of $10{.}5\;\mi$.

\vspace{-0.3pt}
There is also a flux of neutrinos coming to the Earth from outer space. In nature neutrinos are as abundant as photons. Neutrinos are produced during transformations of atomic nuclei ($\beta$-decay, $K$-capture, decays of $\pi$- and $K$-mesons etc.). These processes occur inside the Earth and in the atmosphere, inside the Sun and stars. It is hypothesized that a powerful neutrino flux generated during gravitational collapse of a star carries away a significant part of the released gravitational energy. The energy of natural neutrinos $\EDS_{\nu}$ span a large interval: from relic neutrinos with $\EDS_{\nu}\Simeq 10^4\;\eV$, which fill the Universe with the density of $\Simeq 200\;\cm^{-3}$ according to the Big Bang theory, to the neutrinos born in collisions of cosmic  protons with nuclei of intergalactic space, the resulting charged pions decay into neutrinos with energies of up to $10^{20}\;\eV$.

\vspace{-0.3pt}
A flux of cosmic neutrinos with energy $\EDS_{\nu}\geqslant1\;\GeV$ can reach the value of $1\;\m^{-2}\cdot\s^{-1}\cdot\text{��}^{-1}$. The flux of neutrinos produced in the atmosphere by decaying charged mesons can be hundred times more.

If the Sun luminosity is completely due to the energy of nuclear fusion, the flux of Solar neutrinos on the Earth must be $6{.}5\cdot10^{10}\;\cm^{-2}\cdot\s^{-1}$ since the synthesis of each helium atom is accompanied by emission of two neutrinos.

An important feature of cosmic rays is the energy distribution of particles. Usually the distribution is defined as the flux $I(E)$ of particles which energy is greater than a given energy $E$. The corresponding curve is called a curve of integral spectrum. The integral spectrum of primary cosmic protons is shown in~\refFigure{7_2}. The corresponding curves for other nuclei are very similar.

For the protons with energy $E>5\;\GeV$ the spectrum is well approximated by a function
$$
I(E)\propto E^{-1{.}7}.   \eqMark{7_1}
$$

In general, the energy of primary cosmic rays is absorbed in two steps: first, the energy of a primary particle is converted into the energy of a large number of secondary particles, second, their kinetic energy is spent on ionizing the atmosphere. The secondary cosmic radiation consists of 
%
\fFigure{Integral spectrum of cosmic protons}7_2 {4cm}{3.2cm}{pic/L07_02.eps}
%
hadrons (pions, protons, neutrons etc.), muons, electrons, photons, and neutrinos. These particles are conventionally grouped into the following components: nuclear-active (hadrons), hard (muons), and soft (electrons and photons). Below we discuss these components in some detail. 

Recall that hadrons are the particles which participate in strong interactions. They include protons, neutrons, hyperons, and mesons. Hyperon is a heavy unstable particle which mass is greater than that of a nucleon (proton or neutron), the lifetime is large compared to the nuclear time ($\Simeq 10^{-23}\;\s$), and the spin is half-integer (they are fermions). Hyperons include $\Lambda^0$, $\Sigma^0$, $\Sigma^{\pm}$, $\Xi^0$, $\Xi^{-}$, and $\Omega^{-}$ among others. Mesons are unstable particles which also belong to hadrons, although they are bosons (with integer spin). Mesons include pions ($\pi^0$, $\pi^{\pm}$) and kaons ($K^0$, $K^{\pm}$), to name just a few. For historical reasons muons $\mu^{\pm}$ are also called mesons, although they are leptons.

So, how does generation of various components of the secondary cosmic rays occur? Protons and other nuclei of the primary cosmic radiation collide with atomic nuclei of the Earth atmosphere, thereby splitting them and producing many secondary hadrons, mostly pions. The charged pions, $\pi^{+}$ and $\pi^{-}$ have the half-life $\tau=2{.}5\cdot10^{-8}\;\s$, the neutral pions, $\pi^0$, have the half-life $\tau=0{.}8\cdot10^{-16}\;\s$. The probability to produce a $K$-meson is $5\text{--}10$  times less, hyperons and antiprotons are created with a probability of about $11\%$, while the probability to produce a lepton, either electron or muon, is very small.

The effective cross-section of a multiple particle production process at high energy is almost independent of the energy of colliding particles (it changes less than tens of percent while the energy varies by four orders of magnitude). The approximate constancy of the cross-section resulted in developing the model of <<black spheres>> for describing proton collisions.
%
\hFigure{Interaction of primary proton with the Earth atmosphere}7_3 {7.4cm}{4.3cm}{pic/L07_03.eps}
%
According to this model, when a separation between high-energy hadrons becomes less than the radius of nuclear interaction, a non-elastic process of multiple particle production occurs. The average number of secondary particles (mean multiplicity) slowly grows with energy (like $\ln \EDS$) and does not depend on the type of colliding hadrons. However, the mean multiplicity is much less than the maximal number of secondary particles which can be produced in a single collision. For instance, $70\;\GeV$-protons colliding with target protons (liquid hydrogen) produce only $5\text{--}6$ charged particles while the law of energy conservation allows production of $70$ $\pi$-mesons. Experiment shows that transversal momentum components of the secondary particles are small, $0{.}3\div0{.}4\;\GeV/\s$ on average, and almost constant in a wide energy range. Therefore, the secondary particles form narrow jets along the direction of motion of a primary particle; the greater the energy of the primary particle, the smaller is the width.

Thus the energy of primary particle is mostly converted to the kinetic energy of secondary particles (mostly propagating along the direction of motion of the primary particle).
 
A primary proton can undergo more than $10$ collisions with atmospheric nuclei in which nuclear-active particles are produced. The diagram in~\refFigure{7_3} illustrates a cascade of successive decays and interactions of secondary particles with atmospheric nuclei. 

Charged pions and, partially, kaons produce muons and neutrinos upon their decay. If a charged pion has a high enough energy ($>10^{12}\;\eV$), it does not decay right away because of the time dilation and, together with protons, continues the cascade of interactions as a nuclear-active component of the secondary cosmic rays.

The generation of the nuclear-active component is accompanied by its <<dressing>> in soft and hard components. Neutral pions ($\pi^0$) are the main source of the electron-photon (soft) component.
%
\fFigure{Composition of cosmic rays at various altitudes: \emph{1}~is nuclear-active component; \emph{2}~is electron photon-component; \emph{3}~is muon component; \emph{4}~is the total intensity of cosmic rays; \emph{l}~is thicknness of atmospheric layer measured from the top}7_4 {4.cm}{6.9cm}{pic/L07_04.eps}
%
Because of a small lifetime the pions decay quickly and produce two $\gamma\text{-}$quanta of high energy: $\pi^0\rightarrow\gamma + \gamma$. These $\gamma\text{-}$quanta decay into ($e^{+}e^{-}$) electron-positron pairs when colliding with nuclei, while the ensuing electrons and positrons produce bremsstrahlung $\gamma\text{-}$quanta (braking radiation). (A charged particle is decelerated by the electromagnetic field of nucleus and therefore radiates high-energy photons). These electrons, positrons, and $\gamma\text{-}$quanta will multiply until the ionizing losses due to electrons and positrons (the energy spent for ionizing atoms) become equal to the bremsstrahlung losses. In air this occurs at the energy about $70\;\MeV$.
 
Eventually the charged pions decay as 
$$   
\pi^{\pm}\rightarrow\mu^{\pm}+\nu_{\mu}(\widetilde{\nu}_{\mu}),   \eqMark{7_2} 
$$
thereby producing the hard (muon) component of the secondary radiation.
 
The dependence of intensity of various components of the secondary radiation on a thickness of the traversed atmospheric layer is shown in~\refFigure{7_4}. One can see that the intensity of the nuclear-active component decreases most rapidly and almost vanishes at the sea level ($l\Simeq10^3\;\g/\cm^2$). The electron-photon component dominates at high altitudes but it gets quickly absorbed and becomes less important than the muon component at the sea level.

Thus, the cosmic rays at the sea level are represented by two components which properties are quite different. The particles of one component are strongly absorbed by matter, the absorption coefficient of a particular substance depends on its atomic number $Z$. This component of cosmic radiation is called \textit{soft}. The soft component is almost completely absorbed by a layer of lead $10\div15\;\cm$ thick. The other component is weakly absorbed, and the mass absorption coefficients of substances with different $Z$ are approximately equal. To emphasize its greater penetration ability this component is called \textit{hard}. The soft component includes electrons and photons. The hard component is represented by muons, the particles with mass $207$ times greater than that of the electron.

So, why is the penetration ability of the soft and hard components so different? Ionization losses of particles of the same energy are approximately equal, the possible difference can only be due to the radiation (bremsstrahlung) losses which are inversely proportional to the square of particle mass. This is easy to understand: the radiation power is proportional to the square of the particle acceleration which is inversely proportional to its mass. The latter means that  radiation losses of muons are negligible, which accounts for the difference between the penetration abilities of the hard and soft components.

Another feature of muons is their spontaneous decay (in about $2{.}2\cdot 10^{-6}\;\s$) in electron and two neutrinos (the electron and muon neutrinos):
$$
\mu^{+(-)}\rightarrow e^{+(-)}+\nu_e(\widetilde{\nu}_e)+\widetilde{\nu}_{\mu}(\nu_{\mu}).   \eqMark{7_3}
$$

In a condensed matter the muon energy is lost only due to interaction with atomic shells (ionization losses); for relativistic muons ($E \gg m_{\mu}c^2$) these losses are almost constant and equal to $(dE/dx)\sub{ion}\Simeq2\;\MeV\cdot\cm^2/\g$ in a substance with small $Z$. In gases one should also take into account the spontaneous decay since some muons will decay before they stop, which results in a stronger muon absorption in the atmosphere than in a liquid or a solid matter.  

Indeed, if a muon at rest has the decay time $\tau_0=2{.}2\cdot 10^{-6}\;\s$, a muon traveling at the speed $v$ has the decay time $\tau = \tau_0/\sqrt{1-\beta^2}$. The probability that the muon which traversed the path $L\;\cm$ did not decay is
$$
W(L)=\exp[-L/(\beta\tau c)] = \exp(-L/L\sub{����}),   \eqMark{7_4} $$
where the decay path is
$$
L\sub{dec}=\beta\tau c=\frac{\fbeta\ftau_0 c}{\sqrt{1-\fbeta^2}}=\frac{\fbeta\ftau_0cE_{\mu}}{m_{\mu}c^2}=\frac{\ftau_0p_{\mu}}{m_\mu}.   \eqMark{7_5}
$$

Here $p_{\mu}=m_{\mu}v/ \sqrt{1-\beta^2}$~is the muon momentum, $E_{\mu}=m_{\mu} c^2/ \sqrt{1-\beta^2}$~is its energy, and $m_{\mu}$~is the muon mass.

The decay does not affect the muon energy losses providing $L\sub{dec}\gg L$, where $L$ is determined by the ionization losses,
$$
L=\frac{E_\mu}{(dE/dx)\sub{ion}}\frac{1}{\frho},   \eqMark{7_6}
$$
where $\rho$~is the substance density. Therefore the condition $L/L\sub{dec}\ll1$ implies
$$
\rho\gg\frac{m_{\mu}c^2}{c\fbeta\ftau_0}\frac{1}{(dE/dx)\sub{ion}}.   \eqMark{7_7}
$$

By substituting the numerical values in this equation one can see that the muon decay does not affect their energy losses if 
$$   
\rho \gg 10^{-3}\;\g/\cm^3.   \eqMark{7_8} 
$$

This condition is satisfied by any liquid or solid substance. For air this condition is not met, so the muon absorption in the atmosphere is stronger than in a condensed matter because of the spontaneous decay.
