%translator Russkov, date 27.04.13

\setcounter{Equation}{0} \setcounter{Figure}{0}
\Work
{Measurement of spectrum of $\boldsymbol\alpha$-radiation by means of \\ semiconductor detector}
{Measurement of spectrum of $\boldsymbol\alpha$-radiation by means of \\ semiconductor detector}
{A silicon surface-barrier detector is used to measure spectra of $\alpha$-particles emitted by different radioactive nuclei: $^{226}\mathrm{Ra}$, $^{233}\mathrm{U}$, $^{238}\mathrm{Pu}$, and $^{239}\mathrm{Pu}$. Fine structure of $\alpha$-radiation and a radioactive decay chain of the uranium family are studied.}

\textbf{Properties of $\boldsymbol\alpha$-decay.} Natural (spontaneous) decay of some nuclei accompanied by $\alpha$-particle emission was discovered long ago, at the beginning of the XX-th century, in the course of study of radium and its decay products. In $1903$  E.\,Rutherford showed that $\alpha$-rays, which are easily absorbed and produce a strong ionizing effect, are charged particles and therefore are deflected by electric and magnetic field. In $1909$ the direct experiments of E.\,Rutherford and T.\,Royds proved that they are helium nuclei.

The energy of an $\alpha$-particle escaped from a nucleus can be easily calculated via conservation laws. If a parent nucleus has mass $M_1$ and a daughter nucleus has mass $M_2$, the laws of energy and momentum conservation can be written as
$$
  M_2 c^2=M_1 c^2+m_{\alpha}c^2+T_1+T_{\alpha},
  \eqMark{5_2_1}
$$
$$
  \mathrm\mathbf{p}_1+\mathrm\mathbf p_{\alpha}=0,
  \eqMark{5_2_2}
$$ 
where $T_1$ and $\mathrm\mathbf p_1$ are the kinetic energy and recoil momentum of the daughter nucleus and $T_{\alpha}$ and $\mathrm\mathbf p_{\alpha}$ are the kinetic energy and momentum of $\alpha$-particle.

It is clear that $\alpha$-particle can escape from nucleus if the difference between the rest energy of parent and daughter nuclei is
%
\cFigure{Energy spectrum of $\alpha$-particles emitted by $_{~83}^{212}\mathrm{Bi}$}5_2_1
{6.1cm}{3.5cm}{pic/L05_2_01.eps}
%
greater than the rest energy of $\alpha$-particle. Due to this fact only a heavy nucleus with $A>200$ undergoes $\alpha$-decay: the recoil energy of the nucleus is very low and the kinetic energy of $\alpha$-particle is almost equal to the difference between the rest energies of the parent and daughter nuclei. It is due to this fact that emitted $\alpha$-particle has a strictly determined energy.

However experiment reveals that the energy spectrum of $\alpha$-particles of a variety of $\alpha$-active nuclei consists of several lines with one of them being dominant. The $\alpha$-spectrum of $_{~83}^{212}\mathrm{Bi}$ is shown as an example in~\refFigure{5_2_1}.

The discreteness of energy spectrum and the relative line intensity can be explained as follows: firstly, $\alpha$-particle can be emitted by a nucleus in an exited state (the so-called long range $\alpha$-particle), and secondly, $\alpha$-particle can be emitted from the ground state of parent nucleus into an excited state of daughter nucleus (short range $\alpha$-particle). Two examples of such transitions are shown in~\refFigure{5_2_2}, namely, the decays of $^{238}\mathrm{Pu}$ and $^{212}\mathrm{Po}$.

In the first case ($^{238}\mathrm{Pu}$) $\alpha$-particles of maximal energy correspond to the ground-to-ground state transitions.
%
\cFigure{Alpha-spectra of nuclei $^{238}\mathrm{Pu}$ and $^{212}\mathrm{Po}$}5_2_2
{10.1cm}{5.6cm}{pic/L05_2_02.eps}
%
Besides $\alpha$-decay can occur in an excited state of a daughter nucleus $^{234}\mathrm{U}$ followed by a $\gamma$-transition into the ground state. Decay of $^{212}\mathrm{Po}$ is an example of $\alpha$-particle emission from an excited state. This happens because $^{212}\mathrm{Po}$ is produced in $\beta$-decay of $^{212}\mathrm{Bi}$. A nucleus of $^{212}\mathrm{Po}$ in an excited state can emit $\alpha$-particle or come to the ground state via \mbox{$\gamma$-quantum} emission. Since the half-life of $\alpha$-particle channel is $10^5$ times greater than that of the $\gamma$-decay channel the intensity of long-range $\alpha$-particles turns out to be very low.

Excited states have different spins and parities therefore the difference of angular momenta of parent and daughter nuclei must be carried away by $\alpha$-particle. In other words, $\alpha$-decay is accompanied by a change of the nucleus angular momentum. A simple estimate shows that if $\alpha$-particle has a low momentum $L$, the corresponding angular momentum barrier in a heavy nucleus amounts to the fraction of about $0{.}002L^2A^2=0{.}002l(l+1)$ of the Coulomb barrier. Thus the contribution of angular momentum barrier can be significant only for large $l$.

Usually a heavy nucleus in the ground state is deformed (magic nuclei are exception to this rule). This means that low states are rotational bands, so a parent nucleus decays exactly into these states producing a group of short-range $\alpha$-particles. The energy of rotational levels is
$$
  E\sub{rot}=\frac{\hbar^2}{2\mathcal I}l(l+1).
  \eqMark{5_2_3}
$$

Therefore measuring the fine structure of energy spectrum of $\alpha$-particles one can determine the moment of inertia $\mathcal I$ of a nucleus.

The half-life of an $\alpha$-active nucleus very strongly depends on the energy of emitted particle. The experimental relation (called the Geiger-Nuttall rule) is (see section~IV):
$$
  \lg T_{1/2}=\frac{a}{\sqrt{E_{\alpha}}}+b.
  \eqMark{5_2_4}
$$
According to Eq.~(\refEquation{4_6}) coefficients $a$ and $b$ very weakly depend on the atomic number $Z$.
\vspace{1ex}

\textbf{Decay chain.}
The systematic study of radioactive elements shows that all heavy nuclei with mass number
$A>209$ are unstable with respect to $\alpha$-decay due to increased relative contribution of the Coulomb energy. If the nuclear mass significantly exceeds the boundary value $A=209$, this nucleus decays into a stable one via a chain of sequential decays. Not all decays in the chain are $\alpha$-decays: since in every $\alpha$-decay the mass number $A$ decreases by $4$ and the atomic number $Z$ decreases by two, the portion of neutrons increases and a nucleus becomes unstable with respect to $\beta$-decay. Therefore in a decay chain, or decay series, $\alpha$- and $\beta$-decays alternate.

An example of such a decay chain is the $^{238}\mathrm{U}$-series which is shown in~\refFigure{5_2_3}.

This series begins with $\alpha$-active isotope of uranium $_{~92}^{238}\mathrm{U}$ which  transmutes into $_{~90}^{234}\mathrm{Th}$ with a half-life of $4{.}5\cdot 10^9$ years, etc. Among nuclei of this uranium-series there is an isotope of radium $_{~88}^{226}\mathrm{Ra}$ which decay chain is studied in the experiment. Very soon a pure sample $_{~88}^{226}\mathrm{Ra}$ ($T_{1/2}=1617$ years) becomes contaminated with its daughter products $_{~86}^{222}\mathrm{Rn}$ ($T_{1/2}=3{.}8$ days), $_{~84}^{218}\mathrm{Po}$ ($T_{1/2}=3$ min), and $_{~84}^{214}\mathrm{Po}$ ($T_{1/2}=10\;\s$) which are $\alpha$-active themselves. Therefore measuring $\alpha$-spectrum of radium-$226$we actually observe the $\alpha$-particles emitted by all its daughter products.
%
\cFigure{Chain of radioactive transformations $^{238}\mathrm{U}\rightarrow^{206}\mathrm{Pb}$}5_2_3
{11.1cm}{6.8cm}{pic/L05_2_04.eps}
%
\vspace{1ex}

\textbf{\so{Experimental installation}}\vspace{5pt}

The main part of the installation is a spectrometer of $\alpha$-radiation. The spectrometer consists of three separate parts: a measuring module, a personal computer with a built-in board of analog-to-digital converter (ADC) and evacuation system \textit{ES} of vacuum chamber \textit{VC} with display unit \textit{DU} (see~\refFigure{5_2_4}).

The measuring module contains the following devices:

\begin{Enumerate}{tab}
\Item. the vacuum chamber \textit{VC} containing a specimen holder, a surface-barrier semiconductor detector and a pressure indicator;

\Item. a noiseless preamplifier \textit{PA};

\Item. a spectrometer amplifier \textit{SA} with controls;

\Item. an adjustable block of low-voltage shift \textit{BLVS} for power supply of the detector.
\end{Enumerate}
A vacuum pump maintains the pressure in the measuring chamber less than $10^{-2}\;\tor$. The semiconductor detector registers $\alpha$-particles with energies in the range from $3{.}5$ to $9\;\MeV$, its energy resolution is less than $30\;\keV$ for the $\alpha$-particle energy of $5{.}1\;\MeV$.

In a surface-barrier semiconductor counter the transformation of incident particle energy into electrical pulses takes place in the region of ($p$--$n$)-barrier. The barrier is a thin junction layer between the regions of $p$- and $n$-type conductivity. A particle passing via the barrier layer creates electron-holes pair along its track. The charge carriers are driven by electric field of the ($p$--$n$)-barrier in the opposite directions that generates a current pulse passing through the crystal. The operation principle of a semiconductor detector of charged particles is discussed in detail in~ Appendix II.

%
\cFigure{Block diagram of small-size spectrometer of $\alpha$-rays}5_2_4
{10.9cm}{6.1cm}{pic/L05_2_05.eps}
%

The resolving power of a detector used for spectrometric aims is of special importance, i.e. the width of distribution curve of pulse amplitudes for a precisely constant energy of detected particles. This distribution curve is usually similar to the error curve (Gaussian curve)
$$
  W(U)dU=\frac{1}{\sqrt{2\pi}\sigma}e^{-(U-U_0)^2/(2\sigma^2)}dU.
  \eqMark{5_2_5}
$$
Here $U_0$ is the average pulse amplitude, $U$ is an amplitude, $W(U)dU$ is the probability that for a particle energy $E$ the amplitude of a measured pulse belongs to the interval between $U$ and $U+dU$ and $\sigma$ is a parameter specifying the distribution width (standard deviation).

The distribution~(\refEquation{5_2_5}) is a bell-shaped curve with a maximum at $U=U_0$. The resolving power of spectrometer is determined via the value of $\delta$, i.e. the width of $W(U)$ measured at one-half of the curve height. The following quantity is usually called the energy resolution of spectrometer
$$
  R=\frac{\delta}{U_0}\cdot 100\%.
  \eqMark{5_2_6}
$$

One can easily find the relation between $\delta$ and $\sigma$:
$$
  \delta =2\sqrt{2\ln 2}\,\sigma.
  \eqMark{5_2_7}
$$

One of the main reasons responsible for the dispersion of pulse amplitudes is statistical fluctuation of the number of electron-hole pairs produced by incident particle. The average number of pairs $N$ equals
$$
  N=E/\EDS\sub{av},
  \eqMark{5_2_8}
$$
where $E$ is the energy lost by the particle in the detector, and $\EDS\sub{av}=3{.}6\;\eV$ is the energy required to produce an electron-hole pair. The standard deviation $\sigma$ equals
$$
  \sigma =\sqrt{N}=\sqrt{E/\EDS\sub{av}}.
  \eqMark{5_2_9}
$$

The contribution due to a fluctuation of the number of pairs into the energy resolution equals
$$
  R\sub{fluc}=\frac{\sigma}{N}\cdot100\%=\sqrt{\frac{\EDS\sub{av}}{E}}\cdot 100\%.
  \eqMark{5_2_10}
$$

Another reason of the dispersion is a noise of electric circuits produced by leakage currents due to thermal generation of electron-hole pairs in the barrier layer and the noise of the first amplifying stage. The less the noise brought by measurement circuits, the closer is the energy resolution of the spectrometer to the fluctuation limit determined by Eq.~(\refEquation{5_2_9}).

The ADC board transforms electrical analog pulses into a digital code which is recorded into the computer memory. On the computer screen one observes a dependence of the number of detected pulses on their amplitude, i.e. the energy spectrum of $\alpha$-particles emitted by the source. \vspace{1ex}

\textbf{\so{Directions}}\vspace{5pt}
\begin{Enumerate}{tab}
\Item. To measure the spectrum of $\alpha$-particles emitted by $_{~94}^{239}\mathrm{Pu}$.

\Item. Using the spectrum calibrate the channels in energy units ($\MeV$).

\Item. Calculate the energy resolution of the detector and estimate the contribution due to  fluctuation of the number of produced pairs (see Eq.~(\refEquation{5_2_10})) and due to the noise of electrical circuits.

\Item. Measure the fine structure of the spectrum of $\alpha$-particles emitted by $_{~94}^{239}\mathrm{Pu}$ and determine the energy of the state of $_{~92}^{235}\mathrm{U}$ into which the transition accompanied by emission of short-range particles occurs.

\Item. Measure the spectrum of $_{~88}^{226}\mathrm{Ra}$. Determine the energies of detected particles emitted by parent and daughter nuclei. Plot the dependence of logarithm of half-life on the energy of $\alpha$-particle, the Geiger--Nuttall plot (Eq.~(\refEquation{5_2_4})).

\Item. Using the plot determine the constants $a$ and $b$ and compare them with those specified by Eq.~(\refEquation{4_6}).

\Item. Calculate the intensity of fine structure lines $_{~94}^{239}\mathrm{Pu}$ assuming that it is determined just by the barrier penetrability (see Eq.~(\refEquation{4_4})).
\end{Enumerate}
\vspace{4pt}

\textbf{\so{Additional task}}

\vspace{4pt}
\begin{Enumerate}{tab}
\Item.
Determine the resolving power of the detector at different energies by approximating the spectra of $_{~88}^{226}\mathrm{Ra}$ obtained by Gauss curves (see Eq.~(\refEquation{5_2_5})).

\Item. Approximating the spectrum of $_{~94}^{239}\mathrm{Pu}$ by two Gaussians determine the precise energy of short-range $\alpha$-particles emitted by this isotope.

\Item. Measure the fine structure of $\alpha$-spectrum of $_{~94}^{238}\mathrm{Pu}$. Then using the results obtained and Eq.~(\refEquation{5_2_3}) determine the moment of inertia of a nucleus of $_{~92}^{234}\mathrm{U}$ (the rotating band of this nucleus is the final state of transitions from the parent nucleus (see Eq.~\refFigure{5_2_2})).
\end{Enumerate}

\vspace{-8pt}
\begin{center}\so{\textsf{\small LITERATURE}}\end{center}
\vspace{-4pt}
{\small

1. \textit{Shirokov\;Yu.\;M., Yudin\;N.\;P.} Nuclear physics.\,---\,M.: Science, $1980$. Ch.\;VI, \textsection\textsection\;$1$--$3$; ch.\;IX, \textsection\textsection\;$1$, $2$.

2. \textit{Muhin\;K.\;N.} Introduction to nuclear physics.\,---\,M.: Atompress, $1965$. Ch.\;II, \textsection\;$9$; ch.\;IV, \textsection\;$18$.

3. \textit{Abramov\;A.\;L., Kazansky\;Yu.\;A., Matusevich\;E.\;S.} Theory of experimental methods of nuclear physics.\,---\,M.: Atompress, $1977$. Ch.\;$5$, $7$.

4. \textit{Tsipenyuk\;Yu.\;M.} Principals and methods of nuclear physics.\,---\,M.: Energyatompress, $1993$. \textsection\;$2{.}6$.
}