%translator Svintsov, date 21.04.13

\setcounter{Equation}{0} \setcounter{Figure}{0}
\Work
{Study of inner photoelectric effect in ($\emph{\textbf{p}}$--$\emph{\textbf{n}}$) junction}
{Study of inner photoelectric effect in ($p$--$n$) junction}
{The aim of the experiment is to study photoelectric properties of barrier-layer photovoltaic cell and the dependence of photocurrent and photo-emf on the wavelength of incident light}

Solar batteries are an alternative (with respect to gas, oil, and nuclear energy) source of electric energy. In a solar battery solar energy is converted into electric energy in barrier-layer photovoltaic cells which are basically ($p$--$n$) junctions formed by two thin plates 
%
\fFigure{Structure of standard solar cell}11_6_1 {5.3cm}{3.41cm}{pic/L11_6_01.eps}
%
of $p$- and $n$-type semiconductors (see~\:\refFigure{11_6_1}). The total thickness of the plates is as little as a few tenths of millimeter, the galvanic contact area is about $2$\:cm$^2$.
The plate {\emph{1}} which receives incident light is much thinner than the plate {\emph{2}}, so the light illuminates both plates. The plate surface is covered with metal electrode-layers, the lower electrode is solid and located on the back side of the plate \emph{2}. The upper electrode is made of narrow stripes on the edge of the plate to allow free passage of light to the lower plate. 

Let us consider behavior of illuminated {($p$--$n$)} junction. 
\refFigure{11_6_2} shows the energy diagram of {($p$--$n$)} junction.

An unlit photovoltaic cell has a built-in potential across the {($p$--$n$)} junction. This potential barrier emerges due to diffusion of charge carriers - electrons and holes - through the junction. In equilibrium the junction current is zero. 

Absorption of photons with an energy exceeding the band gap results in creation of electron-hole pairs on both sides of the {($p$--$n$)} junction. It is important to consider the behavior of minority charge carriers (holes in the $n$-type semiconductor and electrons in the $p$-type) in each area, since their density varies in a wide range while the density of the majority carriers on both sides of {($p$--$n$)} junction does not significantly change. Minority charge carriers near the junction are quickly transferred by the junction field and become majority charge carriers on the other side (see~\:\refFigure{11_6_2}).

The junction acts as a drain for the minority carriers generated by light, which creates a concentration gradient, so the minority charge carriers located at a distance of the diffusion length from the junction diffuse to it. This process is supported by carrier diffusion caused by the exponential gradient of carrier density due to optical absorption in the near-surface region.
 
In the base region, i.e. in the region behind the {($p$--$n$)} junction with respect to the illuminated surface, there is a concentration gradient directed from the {($p$--$n$)} junction deep into the base region due to penetration of light. Thus, this diffusion flow and the flow caused by the junction field are opposite. However, the concentration gradient in the base region is negligible, so the minority charge carriers flow to the junction from this region. It is tacitly assumed that the {($p$--$n$)} junction sweeps optically-generated charge carriers at a distance which does not exceed the diffusion length.
%
\hFigure{Energy band diagram of {($p$--$n$)}junction: \emph{�} without illumination; \emph{b} with photoexcitation on both sides of {($p$--$n$)} junction (the voltage $\Delta V$ is equal to the difference between the Fermi levels)}11_6_2 {7.5cm}{3.3cm}{pic/L11_6_02.eps}
%

\vspace{4pt}
While discussing light absorption in a semiconductor we do not take into account the Coulomb attraction of the excited electron-hole pair. Actually the Coulomb attraction may result in the excited state in which an electron and a hole remain bound in a hydrogen-like (or rather positronium-like) state. The ground energy of such an excited state (called exciton) is of the order of 4\,meV, which is much smaller than the band gap width. The characteristic size of exciton is about 15\,nm, which substantially exceeds the interatomic distance. Since exciton is neutral it can move (diffuse) easily through a crystal; when it approaches the {($p$--$n$)} junction it breaks up in the junction electric field with its electron going into the $n$-region and the hole going to the $p$-region.

The excess of minority charge carriers near the {($p$--$n$)} junction reduces the inner potential barrier of the junction $V_D$, as it is shown in~\refFigure{11_6_2}\emph{b}; this can be ascribed to a partial neutralization of the junction bulk charge. This in turn means that the 
%
\fFigure{Current-voltage characteristics of silicon photovoltaic cell at different illumination}11_6_3 {4.9cm}{5.3cm}{pic/L11_6_03.eps}
%
Fermi levels in the bulk semiconductor on both sides of the junction do not coincide any more. The ensued potential difference $\Delta V$ between two regions ($p$-region is positively charged relative to the $n$-region) depends on the resistance of the external circuit of the junction. The highest voltage is obtained at the relevant illumination intensity and open circuit; it can approach the inner potential barrier, i.e. the built-in voltage of the {($p$--$n$)} junction. If the diode is short circuited the current in the circuit is maximal. The current-voltage characteristics of a semiconductor system with a {($p$--$n$)} junction is described by the following equation
 $$
I=I_s\left(e^{eV/kT}-1\right)-I_L , \eqMark{11_6_1}
$$
where  $I$ is the current in the external circuit, $I_s$ is the saturation current of the unlit {($p$--$n$)} junction, $V$ is the external potential difference across the junction, and $I_L$ is the current due to photoexcited charge carriers. A series of curves in~\refFigure{11_6_3} illustrates this dependence.

Each curve corresponds to some value of $I_L$. The curve passing through the origin corresponds to $I_L=0$ (no illumination). The intercept of a current-voltage characteristics and the current axis yields the short-circuit current $I_{\rm sc}$ (i.e. when $R\sub{\rm l}$=0), and the intercept with the voltage axis yields the open-circuit voltage $V_{��}$ ($R\sub{\rm l}=\infty$) of the photoelectric system (see~\refFigure{11_6_4}; the shaded region area equals the power dissipated by the load $R\sub{l}$). Setting $I=0$ in the current-voltage characteristics we obtain the open-circuit voltage
$$
V_{oc}={kT\over e}\left(\ln{I_L\over I_s}+1\right). \eqMark{11_6_2}
$$
Thus, illumination of {($p$--$n$)} junction results in the so-called {\it barrier-layer photoeffect}. The analysis of its mechanism shows that a photo-emf can emerge when there exist nonequilibrium charge carriers opposite in sign to thermally excited carriers.

Obviously, a photo-emf occurs only when the photon energy $\hbar\omega$ exceeds the band gap width $\Delta$ of the cell semiconductor material. Notice that an electron from the valence band is driven to the conduction band by light. In principle, electrons can pass there from donor levels as well but their energies are so low that actually all donor levels are depleted at room temperature, i.e. all donor electrons are already in the conduction band. Similarly, holes appear in the valence 
%
\fFigure{Load part of current-voltage characteristics of barrier-layer photocell}11_6_4 {3.6cm}{4.12cm}{pic/L11_6_04.eps}
%
band. Thus, the maximum efficiency of conversion of light energy to photocurrent is achieved for photons with energy $\hbar\omega\Simeq\Delta$. At higher photon energies the energy excess $\hbar\omega-\Delta$ is converted to heat, which reduces the conversion efficiency 
$$
\eta={P\over W}, \eqMark{11_6_3}
$$
where $W$ is a power of incident light, and $P$ is a power dissipated in the photovoltaic cell load.

Apparently $\eta$ could be enhanced by increasing the cell illumination. However, an increase in the number of electron-hole pairs due to illumination results in a decreased potential barrier at the {($p$--$n$)} junction. The voltage $V$ across the cell electrodes simultaneously increases and the energy dissipated in the external circuit, which is proportional to the voltage $V$, also grows. Since the voltage $V$ on the electrodes cannot exceed the potential barrier of the unlit {($p$--$n$)} junction the voltage $V$ on the electrodes simply stops growing at high illumination. Besides, the power efficiency is the highest when the load resistance is equal to that of the source. The internal photocell resistance changes with illumination and so does $\eta$.

As it was already mentioned, the barrier-layer photovoltaic cells are mostly used in solar batteries. However, only a part of solar radiation spectrum produces electron-hole pairs, and this part certainly depends on the type of semiconductor used. The temperature of the radiating surface of the Sun is approximately 6000\,K, its radiation spectrum agrees well with that of the black body. The maximum in the radiation spectrum is at the frequency $\omega=2.2\cdot10^{15}$\,rad/s, which corresponds to photons with energy $\hbar\omega=1.4$\,eV. The most suitable photovoltaic cell to exploit solar radiation is made of silicon ($\Delta=1.11$\,eV). This material is abundant (sand), the production technology of crystal silicon is developed, and the manufacture is well mastered.

The conversion efficiency of the state-of-the-art solar-batteries is only $\Simeq 20$\,\%. This can be attributed to the losses of solar radiation on a surface recombination, a heat dissipation inside a photocell, and radiation and reflection from the surface.
\vspace{6pt}

\textbf{\so{Directions}}

\vspace{3pt}
The experiment is carried out on the same laboratory installation which is used to study photoconductivity of semiconductors (11.2), only the photoresistor unit is replaced by the photocell unit connected to the digital voltmeter.

1. Install the unit with a sample barrier-layer photovoltaic cell in the output slit of the monochromator.  

2. Connect the photovoltaic cell electrodes to the multimeter terminals and measure the cell dark current. The measurements are taken in the current measurement mode: use terminals  ''com''  and  ''mA''  of the multimeter, press the buttons  ''mA''  and  ''200\,$\mu A$''.

3. Connect the cell electrodes to the voltmeter terminals (measurements are taken in $\mu$A range) and measure the cell dark current.

4. Measure the photocurrent as a function of the light wavelength. Each measurement must be taken after a two-minute exposure. The wavelength is determined by the calibration curve. The desired photocurrent can be found by subtracting the dark current from the current obtained for a given wavelength.

5. Plot the spectral dependence of photocurrent for the specimen studied. For this purpose the current must be normalized by a constant flux of photons. Use the diagram of spectral distribution of photon flux at the installation.

6. Repeat steps 4 and 5 by measuring the voltage across the photocell electrodes with the digital voltmeter (''$\mu$V'' range).

7. Estimate the range of wavelengths of an incandescent lamp �-12 converted into electric energy by the photovoltaic cell.

8. Repeat steps 6 and 7 by measuring the voltage across the photocell electrodes with a multimeter tester. To do measurements use terminals ''com'' and ''V--$\Omega$'', press buttons ''V'' and ''mV''.

\Literat\small

1.\:\emph{Yu.M. Tsypenyuk.} Quantum Micro- and Macrophysics.\,---\,�.: Phismatkniga, 2006.

2.\:\emph{T. Moss, G. Burrel, B. Ellis.} Semiconductor Opto-Electronics.\,---\,�.: Mir, 1976.

3.\:\emph{V.L. Bonch-Bruevich, S.P. Kalashnikov} Semiconductor Physics.\,---\,�.: Nauka, 1979.
\normalsize
