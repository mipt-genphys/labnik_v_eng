%translator Shcherbakov, date 05.11.12

\Chapter {Wave-particle duality}
{Wave-particle duality}
{Wave-particle duality}

\textbf{De Broglie's waves}.
Classical physics describes particle motion and wave propagation as quite different phenomena. A point particle in classical mechanics is a small fraction of matter localized in an infinitely small volume; its motion is governed by Newton's laws. Such a particle can be <<labeled>>, which allows one to separate it from other particles and to trace its path.

Wave is a completely different object. Classical physics describes a wave as a vibration of matter propagating in space. For instance, an acoustic wave in a medium is considered as a propagating vibration of particles of the medium. These vibrations occur at all points of space reached by the wave. Another type of wave, electromagnetic wave, is of quite different kind. Its propagation is governed by the laws of electrodynamics (Maxwell's equations) instead of Newtonian mechanics. However, there are phenomena which are common for all kinds of waves, namely, interference and diffraction. These phenomena are due to phase shifts between interacting waves.

Quantum mechanics, as the theory of microscopic phenomena, eliminated the fundamental difference between waves and particles. The cornerstone of quantum mechanics is wave-particle duality, i.e. the dual nature of particles. Thus electrons are described as particles for a certain set of phenomena (e.g., electron motion in the Wilson cloud chamber used for <<visualization>> of particle paths) whereas for other phenomena (especially for diffraction) electron behavior is entirely described in terms of wave theory.

The concept of <<matter waves>> was originally proposed in $1924$ by Louis de Broglie and soon it was brilliantly confirmed in the electron diffraction experiments. Another spectacular demonstration of the wave-like nature of matter was given in $1927$ by C.\,Davisson and L.\,Germer, and independently by J.\,J.\,Thomson.

On the other hand, electromagnetic radiation, namely, light manifests its particle-like properties in interaction with matter. In 1905 A.\,Einstein explained the photoelectric effect by representing light as a flow of <<corpuscles>>. These <<corpuscles>>, or light quanta, were later named photons. Another convincing demonstration of the particle-like (quantum) properties of light was given by A.\,Compton in $1923$ in his famous X-ray scattering experiments.

The energy of each light quantum is defined as
$$
E=\hbar \omega.   \eqMark{1_1}
$$
Here $\omega$ is the light angular frequency and $\hbar=1{.}05\cdot 10^{-27}\;\erg\cdot\s$ is the Planck constant. This relation was first introduced by M.\,Planck to explain the emission spectra of solids. Photon momentum $\mathbf{p}$ can be written in a similar form using wavevector $\mathbf{k}$ (wavenumber $|\mathbf k |=2\pi/\lambda$, where $\lambda$ is the light wavelength):
$$
\mathbf{p}=\hbar\mathbf{k}.   \eqMark{1_2}
$$

L.\,de Broglie assumed that the above equations for energy and momentum were valid for any particle. He assumed that a micro-particle propagates as if its motion were described by a wave with the wavelength
$$
\lambda =\frac{2\pi}{k}=\frac{2\pi\hbar}{p}=\frac{2\pi\hbar}{mv},   \eqMark{1_3}
$$
in agreement with (\refEquation{1_2}), where $m$ and $v$ are the particle mass and velocity. Since the kinetic energy $E$ of a relatively slow particle is $mv^2/2$, then
$$
\lambda =\frac{2\pi\hbar}{\sqrt{2mE}}.   \eqMark{1_4}
$$

Since all micro-objects (traditionally they are referred to as <<particles>>) can manifest both particle-and wave properties, they can be considered neither as particles nor waves of classical theory. In $1926$ M.\,Born interpreted de Broglie waves in terms of particle wave function $\psi(\mathbf r , t)$, to which he attributed the wave-like behavior. According to his interpretation the square of the absolute value of wave function gives the time-dependent probability density of locating a particle at the point $\mathbf r$.

The wave function of a free particle with a given momentum along $x$-axis is the de Broglie wave:
$$
\psi(x,t)\propto e^{i(px-Et)/\hbar}.   \eqMark{1_5}
$$
In this case $|\psi|^2=\mathrm{const}$.

The Born interpretation of de Broglie's wave rules out its interpretation as a classical matter wave. For example, an electron plane wave should not be understood literally as if the electron were <<spreaded>> over a vast spatial volume. In fact this means that the probability distribution of detecting the electron at any point in space is uniform.

Thus, quantum-mechanical motion of a free particle with mass $m$ and velocity $v$ (i.e., with momentum $p=mv$) can be described as the propagation of a plane monochromatic wave with the wavelength defined by Eq.~(\refEquation{1_3}) in the direction of particle motion. If the particle is moving along $x$-axis, the function $\psi$ can be written as
$$
\psi\propto \cos(kx+\alpha),   \eqMark{1_6}
$$
where $\alpha$ is the initial phase. In general case the direction of wave propagation is specified by wavevector $\mathbf{k}$, and
$$
\mathbf{k}=\mathbf{p}/\hbar,   \eqMark{1_7}
$$
which follows from Eq.~(\refEquation{1_2}). In other words, vector $\mathbf{k}$ always points in the direction of particle motion. Therefore the wavevector of a monochromatic wave related to a free micro-particle is proportional to the particle momentum.

Interaction of a particle with another object (a crystal, a molecule, etc.) affects the particle motion and therefore the wave related to the particle. Such changes are governed by the principles valid for all wave phenomena. Therefore basic geometric features of particle diffraction are very similar (or even the same) to diffraction pattern of any wave including a radiowave propagating near the Earth surface, a sound wave propagating in a solid, a light wave diffraction on grating, X-ray scattering on a crystal, etc.

