%translator Savrov, date 04.05.13

\Preword

In the third edition of the MIPT general physics laboratory guide the spotted errors and types of the second edition are corrected. The third edition contains several new experiments and the necessary  updates taking into account a significant upgrade of the laboratory equipment since the time of the second edition.

The authors are especially grateful to the head of the General Physics Department at MIPT, professor A.~V.~Maksimychev, for his support and useful discussions. 

\vspace{1.0cm} {\sffamily\upshape\bfseries\fontsize{11pt}{12.5pt}\selectfont \noindent FOREWORD TO THE SECOND EDITION} \vspace{0.6cm}

\noindent In the second edition of the general physics laboratory guide for the III-rd year students of  MIPT we corrected the spotted errors and typos of the first edition. This book includes several new experiments developed by our colleagues of the Department of General Physics in the past seven years. The authors are greatly indebted to the head of the General Physics Department at MIPT, professor A.~D.~Gladun, for his support of upgrading the laboratory practicum of the III-rd year. 

\vspace{1.0cm} {\sffamily\upshape\bfseries\fontsize{11pt}{12.5pt}\selectfont \noindent PREFACE} \vspace{0.6cm}

\noindent This book includes the descriptions of experiments of the general physics laboratory of the III-rd year students of MIPT. In accordance with the curriculum these experiments are in the field of quantum physics in its various manifestations: atoms, atomic nuclei, elementary particles, and condensed matter. 

Each particular section of the curriculum -- lecture, seminar, and laboratory experiment -- has its own goal. An experiment is aimed at demonstrating a certain physical phenomenon. However, while the emphasis of the I-st and II-nd year experiments is on experimental methods, measurement instruments, and treatment of results, a III-rd year experiment is closer to a research project. For this reason this laboratory guide is somewhat different from the previous ones. All experiments are grouped into chapters according to the main parts of the curriculum and each chapter begins with a theoretical introduction. An introduction does not duplicate a textbook. It presents the physical principles underlying the phenomena under study, the main equations, and their qualitative analysis. 

There is another reason for including the theoretical introductions. Often a laboratory experiment precedes the lecture and doing an experiment without clear understanding of the phenomenon under study and/or the experimental methods does not make sense. Besides, a lot of issues concerning the experimental equipment (e.g. operation principles of particle detectors, electronics, thermometry, etc.) are not discussed in lectures and seminars and can be found only in specialized textbooks and monographs.

The descriptions of the experiments are also changed. In addition to the mandatory part some experiments contain additional tasks close to research projects. Thus, a lot of laboratory experiments can serve as the presentations chosen for the bachelor exam in general physics. Actually this has been practiced at the Department for years and proved to be very useful. 

More serious are the requirements to treatment of results of the III-rd year experiments. During the first two years MIPT students receive a sufficient experience in the laboratory and a solid mathematical background. The book contains a detailed discussion of the method of least squares, the $\chi^2$-test, Student's t-distribution, and methods of data extrapolation and interpolation. The discussion of statistical methods is also motivated by good programming skills possessed by the III-rd year students. Personal computers have a lot of data treatment software which opens a wide possibility of applying the statistical methods to laboratory experiments.   

Of course this laboratory guide is a successor of the previous handbooks of this kind. In this respect a prominent role of professor L.~L.~Goldin in developing the laboratory practicum at MIPT should be especially mentioned, who was in charge of the laboratory for many years and the chief editor of the previous editions.

Modern versions of the laboratory experiments described in this book were developed due to diligent work of the Department faculty and staff. Although a lot of experiments have concrete authors who designed the original installations, in the course of years the experiments de-facto became a collective work since they have been constantly updated and upgraded. The authors of this edition simply organized and summarized the experience accumulated by faculty and staff of the Department of General Physics at MIPT over the years. Specifically for this edition the experiments in atomic physics were prepared for publishing by A.~S.~Dyakov (1.1, 2.1, 2.2, 3.1), in nuclear physics by Yu.~A.~Samarskii (1.2, 4.1, 4.2, 4.3, 5.1, 5.3, 6.1, 7.1, 7.3), in solid state physics by F.~F.~Igoshin (8.1, 9.1, 10.1, 10.2, 11.1, 11.2, 11.3, 11.4, 11.5, 12.1). Experiments 1.3, 5.2, 5.4, 7.2, and 12.2 developed by Yu.~M.~Tsipenyuk are novel ones. Theoretical introductions to all chapters and Appendices were written by Yu.~M.~Tsipenyuk who also edited the manuscript. The contribution of G.~N.~Freiberg and V.~V.~Tolstikov in developing the laboratory software and upgrading the nuclear experiments should also be mentioned.

The authors are greatly indebted to the head of the General Physics Department at MIPT, professor S.~P.~Kapitza, who was in charge of the Department for 30 years, for his support and discussion of this edition.

